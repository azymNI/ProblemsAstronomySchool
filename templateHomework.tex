\documentclass[a4paper]{article}

\usepackage[spanish]{babel}
\usepackage{graphicx}
\usepackage{amsmath, amssymb}
\usepackage[margin=2cm]{geometry}
\usepackage{fancyhdr}
\usepackage{enumerate}
\usepackage[shortlabels]{enumitem}
\usepackage{parskip}
\usepackage[most]{tcolorbox}
\usepackage[hidelinks]{hyperref}
\usepackage{float}


% cabecera
\pagestyle{fancy}
\fancyhead[l]{Daniel Soto}
\fancyhead[c]{Problemas de Astronomía \#1}
\fancyhead[r]{\today}
\fancyfoot[c]{\thepage}
\renewcommand{\headrulewidth}{0.2pt} % linea horizontal

\newcounter{problema}

\newtcolorbox[use counter=problema]{problema}[1][]{%
    colback=gray!15,
    colframe=gray!60,
    fonttitle=\bfseries,
    title={Problema~\theproblema:},
    boxrule=0.5mm,
    arc=3mm,
    boxsep=5pt,
    left=8pt,
    right=8pt,
    top=6pt,
    bottom=6pt,
    #1
}


\begin{document}

\begin{problema}
    
    \begin{figure}[H]
        \centering
        \includegraphics[width=0.7\textwidth]{TidalForce.png}
        \caption{Fuerza de marea entre la tierra y la luna}
        \label{fig:1}
    \end{figure}

    Una fuerza de marea es una diferencia en la fuerza de gravedad entre dos puntos, como en la Figura \ref{fig:1}

    
    \begin{equation}
    a = \frac{2GMd}{R^3}
    \label{eq:1}
    \end{equation}

    \begin{enumerate}
        \item La ecuación \ref{eq:1} nos permite calcular la aceleraciòn de marea $a$, através de un
        cuerpo de longitud $d$
    \end{enumerate}
    
    La referencia al libro es la siguiente \cite{algebra2}
\end{problema}

\bibliographystyle{plainurl}
\bibliography{bibliografia} % Nombre del .bib

\end{document}
